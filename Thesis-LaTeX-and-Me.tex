\documentclass{beamer}
    \usetheme{Frankfurt}
    \usecolortheme{seahorse}

\usepackage{iftex}
\ifPDFTeX
  \usepackage[utf8]{inputenc}
  \usepackage[T1]{fontenc}
  \usepackage{lmodern}
\else
  \usepackage{fontspec}
  \setmonofont{DejaVu Sans Mono}
  \usepackage{unicode-math}
\fi

\usepackage{amsmath,amssymb,amsfonts}
\usepackage{newunicodechar}
\usepackage{hhline}
\usepackage{menukeys}
\usepackage{fancyvrb}
\usepackage{todonotes}

\newunicodechar{✔}{\ensuremath{\checkmark}}

\title{Thesis,~\LaTeX and me}
\author{Martin Heuschober}
\institute{}


\begin{document}
\begin{frame}
    \titlepage
\end{frame}

\begin{frame}
    \tableofcontents
\end{frame}

\section{WTF}
\label{sec:wtf}

\subsection{}
\begin{frame}{Worum geht's hier??}
    Ich möchte heute ein paar Tipps und Tricks zeigen mit denen man sein Leben mit
    Latex erleichtern kann und sich mehr aufs Beweisen als aufs Tippen
    konzentrieren kann.
\end{frame}

\begin{frame}{Erstmal ein paar generelle Anmerkungen}
    \begin{itemize}
        \item Benutze mehr als eine Datei\\
              z.B. präambel.tex, kapitel1.tex, kapitel2.tex und binde diese in
              diplomarbeit.tex ein
        \item Lable alle Formeln
        \item Lable sections, subsections, chapter
        \item Lable einfach alles was du möglicherweise irgendwann brauchst
        \item Lass deinen Editor alles labeln
        \item Überlasse generell die unangenehmen Aufgaben deinem Editor
    \end{itemize}
\end{frame}

\begin{frame}{Recherche ist auch notwendig}
    Nur geh' ich ungern in die Bibliothek um zu sehen, dass das Buch (a) nicht
    da ist, (b) doch nicht richtig ist.\\
    Deswegen liebe Leute nutzt das Internet
    \begin{itemize}
        \item \url{http://libgen.info}
        \item \url{http://bookfi.com}
        \item \url{http://books.google.com}
        \item \url{http://http://scholar.google.com}
    \end{itemize}
\end{frame}

\begin{frame}{aber auch für LaTeX ist das Internet gut}
    \begin{itemize}
        \item \url{http://www.google.com}
        \item \url{http://tex.stackexchange.com}\\
            registrieren ist notwendig aber das System ist s-u-p-e-r
        \item \url{http://stackoverflow.com}\\
            eher für programmieren, aber das kommt ja auch manchmal vor z.B.~in
            einer Numerikvorlesung (Matlab/Octave), Mathematica (ist oft
            nützlich), Algorithmen und Datenstrukturen (Java).
    \end{itemize}
\end{frame}

\begin{frame}{und für Mathematik auch!}
    \begin{itemize}
        \item \url{http://math.stackexchange.com}
        \item \url{http://mathoverflow.net}
    \end{itemize}
\end{frame}

\section{Editors}
\label{sec:editors}

\subsection{}
\begin{frame}{Welches Betriebssystem, welcher Editor}
    \begin{center}
        \begin{tabular}{c|ccc}
                         & \textbf{Windows} & \textbf{Mac} & \textbf{Linux} \\ \hhline{=|===}
            TexMaker     & ✔                & ✔            & ✔              \\
            gVim         & ✔                & ✔            & ✔              \\
            emacs        & ✔                & ✔            & ✔              \\
            Eclipse      & ✔                & ✔            & ✔              \\ \hline
            TexShop      &                  & ✔            &                \\ \hline
            WinEDT       & ✔                &              &                \\
            TeXnicCenter & ✔                &              &                \\ \hline
            gEdit        &                  &              & ✔              \\
            gummi        &                  &              & ✔              \\
            kate         &                  &              & ✔
        \end{tabular}
    \end{center}
\end{frame}

\begin{frame}{Was sollte ein Editor können?}
    \begin{enumerate}
        \item Syntax highlighting
        \item AutoCompletion, automatisches einfügen von Text\\
              z.B. \$ $→$ \$|\$\\
              z.B. $\{$ $→$ $\{|\}$\\
              wobei $|$ die position des Cursors sein sollte
        \item Snippets, soll heißen mit <Tab> o.Ä. ausgelöste Ersetzung von Text
            z.B. \texttt{pac}\keys{\tab} $→$ \texttt{\backslash usepackage\{|\}}
        \item Rechteckauswahl - angenehm fürs schön machen von Tabellen und Co.
        \item Auto Alignment wieder fürs schön machen von Gleichungen und
              Tabellen
    \end{enumerate}
\end{frame}

\begin{frame}{Was sollte ein Editor können?}
    \begin{enumerate}\setcounter{enumi}{5}
        \item text-wrapping = automatischer Zeilenumbruch
        \item synctex = synchronisation von der aktuellen bearbeiteten Stelle im
              Quellcode mit der Stelle im PDF-Output
        \item Suchen und Ersetzen besser noch mit Regular Expressions!
        \item Automatische Einrückung (indentation) von Text
        \item Schnelles Kompilieren und bearbeiten von Errors und Warnings
    \end{enumerate}
\end{frame}

\begin{frame}{Der Apfel und seine Eigenheiten}
    \begin{itemize}
        \item Keyboard - \backslash, $\{$, $\}$ \$, $[$, $]$ sind ungemütlich zu
            tippen, da kann AutoCorrection behilflich sein\\
            z.B. üü $→$ \backslash, öö $→$ $\{$
        \item TeXshop kann AutoCompletion und AutoCorrection wobei Completion schon
            ein wenig gefinkelt zum selber basteln ist
    \end{itemize}
\end{frame}

\begin{frame}{Das Fenster zur Welt}
    Ich muss zugeben mit MS Windows wenig Erfahrung zu haben, deshalb kann ich
    nichts zu TeXnicCenter sagen und würde eher Eclipse mit dem Plugin TeXlipse
    empfehlen, da TexMaker zwar schon okay ist aber Eclipse weitaus mächtiger
    ist.
\end{frame}

\begin{frame}{Pinguin und Teufelchen}
    Die Linux/Unix Welt bietet viele Möglichkeiten abgesehen von Werkzeugen aus
    den alten Tagen als man Computer noch in Nullen und Einsen programmiert hat,
    gibt es schon einige neue Editoren ‘gummi’ und ‘gEdit’ sind zwei Beispiele
    aus der Gnome-Ecke, ‘Kile’ und ‘Kate’ aus der KDE-Ecke.\\

    Mein Favorit ist ‘Vim’, da ich mich doch recht lange damit
    auseinander gesetzt habe. ‘gEdit’ hat eine recht ansehnliche Liste an
    Plugins die den Editor benutzbar machen. ‘gummi’ ist eine relativ neue
    Entwicklung, hat integrierte Vorschau, snippets aber keine Dollar-/Klammervervollständigung
    sieht aber vielversprechend aus.
\end{frame}

\begin{frame}{DEMO}
    \begin{center}
        \Huge DEMO
    \end{center}
\end{frame}

\section{Usepackages}
\label{sec:usepackages}

\subsection{}
\begin{frame}{Wissenswertes über LaTeX}
    \begin{itemize}
        \item ACHTUNG usepackages können untereinander in Konflikt stehen und es kommt
              noch besser, die REIHENFOLGE in denen man usepackages einbindet ist
              relevant!
        \item  don't use \$\$ ist veraltet
        \item  use bibtex
    \end{itemize}
\end{frame}

\begin{frame}[fragile]{Zum organisieren}
    \footnotesize
    \begin{verbatim}
\usepackage{todonotes}
\end{verbatim}
    \normalsize
    fügt nette Randnotizen und ein Verzeichnis derer ein.
\end{frame}

\begin{frame}[fragile]{Zum organisieren}
    \footnotesize
    \begin{verbatim}
\usepackage[notcite,notref]{showkeys}
\renewcommand*\showkeyslabelformat[1]{\tiny(#1)\normalsize}
    \end{verbatim}
    \normalsize
fügt Labelnamen an den gelabeleten Stellen ein, aber f***s up amsthm
\end{frame}

\begin{frame}[fragile]{Zum organisieren}
    \footnotesize
    \begin{verbatim}
\usepackage{environ} % just removes the proofs
\NewEnviron{killcontents}{}{}
\let\proof\killcontents
\let\endproof\endkillcontents
    \end{verbatim}
    \normalsize
versteckt die Beweise angenehm zum neu Strukturieren der Arbeit
\end{frame}

\begin{frame}[fragile]{Layout testing - aka irgendwas schaut komisch aus}
    \begin{verbatim}
        \usepackage{showframe}
    \end{verbatim}
\end{frame}

\begin{frame}[fragile]{The math typesetting stuff}
    \footnotesize
\begin{verbatim}
\usepackage[sumlimits,intlimits]{amsmath}
    \DeclareMathOperator{sec}{\sec} % sectional curvature
\end{verbatim}
\pause
\begin{verbatim}
\usepackage{amsfonts,amssymb,bbm,wasysym,mathrsfs}
\end{verbatim}
\pause
\begin{verbatim}
\usepackage{mathtools}
    \mathtoolsset{showonlyrefs}
    % zeigt nur die verwendeten Referenzen an
\end{verbatim}
\pause
\begin{verbatim}
\usepackages{amsthm}
    \renewcommand{\thmnumber}{\arabic{section}.\thenumber}
    \newtheorem{thm}{Theorem}[chapter]
    \newtheorem{dfn}[thm]{Definition} % def is already taken
\end{verbatim}
\end{frame}

\begin{frame}[fragile]{Bunt ist toll und Bilder auch}
\begin{verbatim}
\usepackage{graphicx}
\usepackage{tikz}
\usepackage{xcolor}
\usepackage{color} -- for named colors
  \definecolor{grey}{rgb}{0.65, 0.65, 0.65}
\end{verbatim}
Besonders interessant ist hier das Paket ‘tikz’ das es erlaubt zu Zeichnen!
\end{frame}

\begin{frame}[fragile]{sonst noch Hilfreiches}
    \begin{verbatim}
\usepackage{hyperref}
    \end{verbatim}
    für Links im Dokument.
    \pause
    \begin{verbatim}
\usepackage{enumerate}
    \end{verbatim}
    erlaubt es enumerate Umgebungen zu verändern.
    \pause
    \begin{verbatim}
\usepackage{lipsum}
    \end{verbatim}
    für Blindtext.
\end{frame}

\begin{frame}[fragile]{sonst noch Hilfreiches}
    \begin{verbatim}
renewcommand{\thechapter}{\Roman{chapter}}
\thesection, \thesubsection, \theequation
    \end{verbatim}
    Um die Art der Nummerierung zu verändern.
\end{frame}

\begin{frame}[fragile]{sonst noch Hilfreiches}
    Und zu guter Letzt der Styleguide.
    \begin{verbatim}
\usepackage[l2tabu,orthodox]{nag}
\usepackage{fixltx2e}
\usepackage[all,error]{onlyamsmath}
    \end{verbatim}
\end{frame}

\begin{frame}[fragile]{Für die Typesetting-Nerds ;-)}
    \begin{verbatim}
\usepackage{unicode-math}
    \end{verbatim}
    Nur in Verbindung mit LuaTex/XeTeX erlaubt es Formeln in Unicode setzen
    \begin{verbatim}
                \[
                    ∑_{i=1}ⁿαᵢ
                \]
    \end{verbatim}
        \[
            ∑_{i=1}ⁿαᵢ
        \]
\end{frame}

\section{Version Control}
\label{sec:version_control}

\subsection{}

\begin{frame}{Problem - History - mehrere Rechner}
    Heutzutage hat man gerne einen oder mehrere Laptop/Computer und schreibt
    hier auf der Uni am Laptop, zu Hause am PC und man hat nicht immer Internet.
    Oder man arbeitet mit mehreren Leuten an einem Projekt. Die selben Probleme
    gibt es seit Jahren im Business Software Development.

    Ein weiteres Problem ist die Geschichte; ich hätte gerne die Version von
    letzter Woche meiner *arbeit wiederhergestellt, aber wie speichere ich das
    ganze sinnvoll.
\end{frame}
\begin{frame}{Die Tools der Wahl heutzutage (stand 2014) sind:}
    Git oder Mercurial, so genannte distributed version control systems.
    \begin{itemize}
        \item \url{http://git-scm.com/}
        \begin{itemize}
            \item \url{http://try.github.io}
        \end{itemize}
        \item \url{http://mercurial.selenic.com/}
        \begin{itemize}
            \item \url{http://hginit.com/}
        \end{itemize}
    \end{itemize}
    Ich werde mich hier auf ‘git’ beziehen, weil ich das kann. Es gibt online
    für beide Systeme tutorials.
\end{frame}

\begin{frame}[fragile]{DEMO}
    \begin{center}
        \Huge DEMO
    \end{center}
\end{frame}

\end{document}
